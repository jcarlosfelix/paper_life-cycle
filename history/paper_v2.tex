% This is samplepaper.tex, a sample chapter demonstrating the
% LLNCS macro package for Springer Computer Science proceedings;
% Version 2.20 of 2017/10/04
%
\documentclass[runningheads]{llncs}
%
%
\usepackage{graphicx}
% Used for displaying a sample figure. If possible, figure files should
% be included in EPS format.
%
% If you use the hyperref package, please uncomment the following line
% to display URLs in blue roman font according to Springer's eBook style:
% \renewcommand\UrlFont{\color{blue}\rmfamily}

\begin{document}
%
\title{(Biological) Life-Cycle Algorithm\thanks{Supported by organization x.}}
%
%\titlerunning{Abbreviated paper title}
% If the paper title is too long for the running head, you can set
% an abbreviated paper title here
%
\author{First Author\inst{1}\orcidID{0000-1111-2222-3333} \and
Second Author\inst{2,3}\orcidID{1111-2222-3333-4444} \and
Third Author\inst{3}\orcidID{2222--3333-4444-5555}}
%
\authorrunning{F. Author et al.}
% First names are abbreviated in the running head.
% If there are more than two authors, 'et al.' is used.
%
\institute{Princeton University, Princeton NJ 08544, USA \and
Springer Heidelberg, Tiergartenstr. 17, 69121 Heidelberg, Germany
\email{lncs@springer.com}\\
\url{http://www.springer.com/gp/computer-science/lncs} \and
ABC Institute, Rupert-Karls-University Heidelberg, Heidelberg, Germany\\
\email{\{abc,lncs\}@uni-heidelberg.de}}
%
\maketitle              % typeset the header of the contribution
%
\begin{abstract}
%The abstract should briefly summarize the contents of the paper in 150--250 words.

% ----------------------------------
% [History Drafts]
% [Draft 1] This paper is a proposal for a nature-inspired algorithm that seeks to solve the basic GA problem OneMax. This is a distributed algorithm in all its conception, where the idea is that its mechanism can operate from different containers (or equipment) independently, eliminating waiting times between processes. This algorithm emphasizes the use of cloud computing available resources. Its mechanism is composed of birth, growth, reproduction, and death, where each one of them works concurrently and asynchronously on a population that evolves constantly. This strategy takes a similar focus as the study of bacteria growth in microbiology, where using a microscope, we can observe and analyze the evolution of a population over time. This implementation also can be useful, as an introductory example to the native cloud computing programming, inspired by the GA algorithm operation mechanics. One difference from the GA is in the reproduction mechanism, which is flexible to work in parallel with several strategies simultaneously, for example, tourney selection, random selection of couples in the population, and couple selection with a random mating individual outside of the population.
% ----------------------------------

Several bio-inspired algorithms use population evolution as analogies of
nature. In this paper, we followed a strategy that takes a similar approach as
the study of bacteria growth in microbiology, where using a microscope, we can
observe and analyze the evolution of a population over time. We present an
algorithm inspired by the biological life-cycle of animal species, which
consists of several stages: birth, growth, reproduction, and death. As in
nature, we intend to execute all these stages in parallel and asynchronously on
a population that evolves constantly.

In this paper, we present a novel distributed nature-inspired algorithm for
solving optimization problems. From the ground up, we designed the algorithm as
a cloud-native solution using the cloud available resources to divide the
processing work-load, among several computers or running the algorithm as a
cloud service. The algorithm works concurrently and asynchronously on a
constantly evolving population, using different computers (or containers)
independently, eliminating waiting times between processes.

This algorithm seeks to imitate the natural life cycle, where new individuals
are born at any moment and mature over time, where they age and suffer
mutations throughout their lives. In reproduction, couples match by mutual
attraction, where they may have offspring. Death can happen to everyone: from a
newborn to an aged adult, where the individual's fitness will impact their
longevity. As a proof-of-concept, we implemented the algorithm with Docker
containers by solving the OneMax problem comparing it with a basic GA
(sequential) algorithm, where it showed favorable and promising results.

\keywords{First keyword  \and Second keyword \and Another keyword.}
\end{abstract}
%
%
%
\section{First Section}
\subsection{A Subsection Sample}
Please note that the first paragraph of a section or subsection is
not indented. The first paragraph that follows a table, figure,
equation etc. does not need an indent, either.

Subsequent paragraphs, however, are indented.

\subsubsection{Sample Heading (Third Level)} Only two levels of
headings should be numbered. Lower level headings remain unnumbered;
they are formatted as run-in headings.

\paragraph{Sample Heading (Fourth Level)}
The contribution should contain no more than four levels of
headings. Table~\ref{tab1} gives a summary of all heading levels.

\begin{table}
\caption{Table captions should be placed above the
tables.}\label{tab1}
\begin{tabular}{|l|l|l|}
\hline
Heading level &  Example & Font size and style\\
\hline
Title (centered) &  {\Large\bfseries Lecture Notes} & 14 point, bold\\
1st-level heading &  {\large\bfseries 1 Introduction} & 12 point, bold\\
2nd-level heading & {\bfseries 2.1 Printing Area} & 10 point, bold\\
3rd-level heading & {\bfseries Run-in Heading in Bold.} Text follows & 10 point, bold\\
4th-level heading & {\itshape Lowest Level Heading.} Text follows & 10 point, italic\\
\hline
\end{tabular}
\end{table}


\noindent Displayed equations are centered and set on a separate
line.
\begin{equation}
x + y = z
\end{equation}
Please try to avoid rasterized images for line-art diagrams and
schemas. Whenever possible, use vector graphics instead (see
Fig.~\ref{fig1}).

\begin{figure}
\includegraphics[width=\textwidth]{fig1.eps}
\caption{A figure caption is always placed below the illustration.
Please note that short captions are centered, while long ones are
justified by the macro package automatically.} \label{fig1}
\end{figure}

\begin{theorem}
This is a sample theorem. The run-in heading is set in bold, while
the following text appears in italics. Definitions, lemmas,
propositions, and corollaries are styled the same way.
\end{theorem}
%
% the environments 'definition', 'lemma', 'proposition', 'corollary',
% 'remark', and 'example' are defined in the LLNCS documentclass as well.
%
\begin{proof}
Proofs, examples, and remarks have the initial word in italics,
while the following text appears in normal font.
\end{proof}
For citations of references, we prefer the use of square brackets
and consecutive numbers. Citations using labels or the author/year
convention are also acceptable. The following bibliography provides
a sample reference list with entries for journal
articles~\cite{ref_article1}, an LNCS chapter~\cite{ref_lncs1}, a
book~\cite{ref_book1}, proceedings without editors~\cite{ref_proc1},
and a homepage~\cite{ref_url1}. Multiple citations are grouped
\cite{ref_article1,ref_lncs1,ref_book1},
\cite{ref_article1,ref_book1,ref_proc1,ref_url1}.
%
% ---- Bibliography ----
%
% BibTeX users should specify bibliography style 'splncs04'.
% References will then be sorted and formatted in the correct style.
%
\bibliographystyle{splncs04}
\bibliography{bib/bibliografia.bib}
%
\end{document}
